\documentclass[12pt]{report}

\usepackage[bahasa]{babel}

\usepackage{tgtermes}
\usepackage[T1]{fontenc}
%\usepackage{libertinust1math}

%\usepackage{fontspec}
%\setmainfont{TeX Gyre Termes}[
%	Numbers = OldStyle,
%5	Kerning = On
%	]z

\usepackage{tikz}

\usepackage[all]{nowidow}

\usepackage{expex}
   \lingset{
       belowglpreambleskip=-0.2ex,% shrinks the vertical space between the preamble and the top gloss line
       everyglpreamble=\it,
       %everygla=,% removes the default italic formatting of the top gloss line
       aboveglftskip=-0.2ex,% shrinks the vertical space between the aligned lines and the free translation line
       interpartskip=0pt,% vertical space between parts of examples
       glspace=!0pt plus .2em,% improves line breaking by increasing the maximum horizontal space between aligned words
       glrightskip=0pt plus .5\hsize,% improves line breaking by increasing the maximum horizontal space between end of line and the right margin
       aboveexskip=1ex plus .4ex minus .4ex,% vertical space above examples
       belowexskip=1.5ex plus .4ex minus .4ex% vertical space below examples
   }

\usepackage{titlesec}
\titleformat{\chapter}[hang]
{\huge\bfseries}
{\thechapter}
{1em}
{}

\linespread{1.5}

%shorthands
\def\ts{t͡s}
\def\dz{d͡ʒ}
\newcommand\gl[1]{\textsc{\MakeLowercase{#1}}}
\newcommand\krv[1]{\emph{#1}}

%basic-infos
\title{Hiyang}
\author{Mallory Cessair}

\begin{document}
 
\maketitle

\chapter{Melihat Dunia Kredan}
\section{Hiyang: Orang Bertaring}
Hiyang merupakan rekabuana masyarakat yang berada di dunia fiksi bernama Kredan, atau Karedan dalam bahasa Kihaga yang dituturkan oleh orang Hiyang. Rekabuana Kredan diciptakan oleh satu orang penulis, yang disebut sevenorbs, yang kemudian dikembangkan lagi sehingga Kredan saat ini diciptakan dari 2 orang penulis. Hiyang dan segala rekabuana-nya merupakan hasil ciptabuana dari penulis kedua mengenai Kredan, yakni Cessair. Ide ciptabuana ini, mulai dari bahasa hingga masyarakat diambil dengan mencontoh masyarakat dan bahasa orang austronesia yang berada di Pulau Sumatera. Pada ciptabuana bahasa untuk orang-orang hiyang, sebagaian besar meniru pada bahasa asli yakni bahasa batak.


\end{document}

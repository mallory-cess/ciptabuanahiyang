\documentclass[12pt]{report}

\usepackage[bahasa]{babel}

\usepackage{tgtermes}
\usepackage[T1]{fontenc}
%\usepackage{libertinust1math}

%\usepackage{fontspec}
%\setmainfont{TeX Gyre Termes}[
%	Numbers = OldStyle,
%5	Kerning = On
%	]z

\usepackage{tikz}

\usepackage[all]{nowidow}

\usepackage{expex}
   \lingset{
       belowglpreambleskip=-0.2ex,% shrinks the vertical space between the preamble and the top gloss line
       everyglpreamble=\it,
       %everygla=,% removes the default italic formatting of the top gloss line
       aboveglftskip=-0.2ex,% shrinks the vertical space between the aligned lines and the free translation line
       interpartskip=0pt,% vertical space between parts of examples
       glspace=!0pt plus .2em,% improves line breaking by increasing the maximum horizontal space between aligned words
       glrightskip=0pt plus .5\hsize,% improves line breaking by increasing the maximum horizontal space between end of line and the right margin
       aboveexskip=1ex plus .4ex minus .4ex,% vertical space above examples
       belowexskip=1.5ex plus .4ex minus .4ex% vertical space below examples
   }

\usepackage{titlesec}
\titleformat{\chapter}[hang]
{\huge\bfseries}
{\thechapter}
{1em}
{}

\linespread{1.5}

%shorthands
\def\ts{t͡s}
\def\dz{d͡ʒ}
\newcommand\gl[1]{\textsc{\MakeLowercase{#1}}}
\newcommand\krv[1]{\emph{#1}}

%basic-infos
\title{Hiyang}
\author{Mallory Cessair}

\begin{document}
 
\maketitle

\chapter{Melihat Dunia Kredan}
\section{Sejarah}
Pada pertengahan tahun 2020, ide mengenai ciptabuana muncul dalam pemikiran penulis yang dipicu oleh diskusi bersama dengan seorang sahabat dan saudara. Dimulai dari sebuah cerita mengenai dunia bernama Kredan, yang merupakan proyek hobi yang dimulai sejak dari masa remaja sahabat penulis. Dari cerita ini, kemudian penulis terpicu untuk ikut berkontribusi pada proyek hobi tersebut. Dari hobi satu orang, menjadi proyek bersama. 

Dimulai dengan memilih cerita mengenai suatu ras yang disebut dengan Hiyang, yang dalam bahasareka Kihaga bermakna orang bertaring. Kemudian berlanjut menjadi sebuah ciptabuana kompleks, dari ras, bahasa, agama, dan flora dan fauna yang ada di Kredan. Proyek bersama ini menjadi salah satu nafas kegiatan bersama kami. Menjadi kegiatan yang tidak hanya mengisi waktu luang kami, namun juga menuangkan ide-ide dan imajinasi di kepala kami.

\section{Mengenal Hiyang}
Hiyang (Kihaga:Orang Bertaring) merujuk pada kelompok orang pendatang yang menetap dan tinggal di daratan Kahlih di bagian ujung timur Kredan. Hiyang menggunakan bahasa Kihaga sebagai bahasa sehari-hari mereka. Masyarakat Hiyang didominasi sebagai penganut agama Abarik, yang mempercayai adanya satu pencipta dengan banyak rupa. Orang-orang Hiyang dapat ditemukan hampir diseluruh bagian Kredan, terutama di daerah Lejar. Beberapa orang Hiyang menetap di bagian daerah Leuwaner yang banyak berprofesi sebagai pedagang keliling antar daerah.

\subsection{Sejarah}
Kedatangan orang-orang Hiyang tidak tertulis secara pasti
%Hiyang merupakan rekabuana masyarakat yang berada di dunia fiksi bernama Kredan, atau Karedan dalam bahasa Kihaga yang dituturkan oleh orang Hiyang. Rekabuana Kredan diciptakan oleh satu orang penulis, yang disebut sevenorbs, yang kemudian dikembangkan lagi sehingga Kredan saat ini diciptakan dari 2 orang penulis. Hiyang dan segala rekabuana-nya merupakan hasil ciptabuana dari penulis kedua mengenai Kredan, yakni Cessair. Ide ciptabuana ini, mulai dari bahasa hingga masyarakat diambil dengan mencontoh masyarakat dan bahasa orang austronesia yang berada di Pulau Sumatera. Pada ciptabuana bahasa untuk orang-orang hiyang, sebagaian besar meniru pada bahasa asli yakni bahasa batak.


\end{document}
